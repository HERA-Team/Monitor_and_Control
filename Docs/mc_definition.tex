\documentclass{article}
\usepackage{float,amsmath}
\usepackage{graphicx}
\usepackage{color}
\usepackage[letterpaper,margin=1in]{geometry}
\usepackage{hyperref}

\usepackage{outlines}
\usepackage{enumitem}
\setenumerate[1]{label=\arabic*.}
\setenumerate[2]{label=\alph*.}
\setenumerate[3]{label=\arabic*.}
\setenumerate[4]{label=\roman*.}

\newcommand{\mc}{M\&C}


\begin{document}

\author{HERA Team}
\title{HERA Monitor and Control Subsystem Definition}
\maketitle

\section{Introduction}
HERA is an international experiment to detect and characterize the Epoch of
Reionization (EOR).  The telescope is located at the South African SKA site in
the Karoo Astronomy Reserve.  This note summarizes Monitor and Control (\mc) subsystem for HERA.

Monitor and Control provides a common place for logging of metadata and messages. The \mc\ system is built around a database with a well documented table schema and a software layer to provide a simple developer framework. It will also include various online daemons for monitoring things, and both a front end web-based user interface and a command-line interface to support analysis code.

\section{Requirements}
\begin{outline}[enumerate]
	\1 Ability to fully reconstruct the historical state of the system.
	\1 All interactions between subsystems must go through or be logged by \mc.
		\2 Both subsystems in an interaction are responsible for logging communications to \mc.
		\2 Subsystems in an interaction are responsible for logging communications to \mc.
	\1 Operational metadata (e.g. temperatures, correlator bit occupancies) must be logged to \mc.
	\1 High availability (\mc\ must not limit uptime of telescope).
	\1 \mc is a provider of information about observations to end-users and must be available to them
\end{outline}

\section{Design Specification}
\begin{outline}[enumerate]
	\1 SQL database
		\2 DB Design principle: every logical sub group has a group of tables.  One adds tables to do more things. E.g. different versions of subsystems add new tables. Operations reference which tables they use.
		\2 This document (and appendices) will contain all table definitions.
		\2 Use careful dB design to avoid duplicated data, make table links/data relationships clear, use many-to-one and many-to-many links.
		\2 Transactions must be used to ensure DB integrity.
		\2 Must be mirrored in some fashion to observer locations.
	\1 At least one SW interface layer will be provided.
		\2 It�s not required to interact with \mc.
		\2 Must support relational db (i.e. multiple column primary and foreign keys) and transactions.
	\1 Hardware
		\2 LOM capabilities
		\2 Multi-teraByte mirrored disk RAID
		\2 Backup machine available on site
\end{outline}

\section{Table Definitions}
Primary keys are bold, foreign keys are italicized.

\subsection{Observations}
\textbf{\large{hera\_obs}}: This is the primary observation definition table.
\begin{center}
 \begin{tabular}{| p{4cm} | p{2cm} | p{10cm} |} 
 \hline
 column & type & description \\ [0.5ex]  \hline\hline
 \textbf{obsid} & long integer & start time in floor(GPS) seconds. GPS start adjusted to be within 1 second of LST to lock observations to LST for the night \\ \hline
 starttime & double & start time in gps seconds. The start time to full accuracy of the beginning of integration of first visibility \\\hline
 stoptime & double & stop time in gps seconds. The stop time to full accuracy of the end of integration of last visibility \\\hline
 jd\_start & double & start time in JD. Calculated from starttime, provides a quick way to filter on JD times. \\\hline
 lst\_start\_hr & double & decimal hours from start of sidereal day. Calculated from starttime, provides a quick search for matching LSTs \\\hline
 \end{tabular}
\end{center}

\subsection{Common tables}
\textbf{\large{server\_status}}: Common table structure for server status info. Same columns to be used in subsystem-specific instances of this table.
\begin{center}
 \begin{tabular}{| p{4cm} | p{2cm} | p{10cm} |} 
\hline
 column & type & description \\ [0.5ex]  \hline\hline
 \textbf{hostname} & string &  name of server \\ \hline
 \textbf{mc\_time} & long & time report received by \mc\ in floor(gps seconds) \\ \hline
 ip\_address & string & IP address of server (how should we handle multiples?) \\\hline
mc\_system\_timediff & float & difference between \mc\ time and time report sent by server in seconds \\\hline
num\_cores & integer & number of cores on server \\\hline
cpu\_load\_pct & float & CPU load percent = total load / num\_cores, 5 min average  \\\hline
uptime\_days & float & server uptime in days  \\\hline
memory\_used\_pct & float & percent of memory used, 5 min average  \\\hline
memory\_size\_gb & float & amount of memory on server in GB \\\hline
disk\_space\_pct & float & percent of disk used  \\\hline
disk\_size\_gb & float & amount of disk space on server in GB \\\hline
network\_bandwidth\_mps & float & Network bandwidth in MB/s, 5 min average. Can be null \\\hline
\end{tabular}
\end{center}

\textbf{\large{subsystem\_errors}}: Subsystem errors/issues
\begin{center}
 \begin{tabular}{| p{4cm} | p{2cm} | p{10cm} |} 
\hline
 column & type & description \\ [0.5ex]  \hline\hline
\textbf{id} & long & auto-incrementing error id\\ \hline
time & long & error report time in floor(gps seconds)\\ \hline
subsystem & string & name subsystem with error (e.g. `librarian', `rtp')\\ \hline
mc\_time & long & time report received by \mc\ in floor(gps seconds) \\ \hline
severity & int & integer indicating severity level, 1 is most severe \\ \hline
log & text & TBD on format, either a message or a file with the log \\ \hline
\end{tabular}
\end{center}

\subsection{RTP Tables}
\textbf{\large{rtp\_server\_status}}: RTP version of the server\_status table\\

\textbf{\large{rtp\_status}}: High level RTP status
\begin{center}
 \begin{tabular}{| p{4cm} | p{2cm} | p{10cm} |} 
\hline
 column & type & description \\ [0.5ex]  \hline\hline
\textbf{time} & long & status time in floor(gps seconds)\\ \hline
status & string & status string, options TBD (might become an enum) \\\hline
event\_min\_elapsed & float & minutes elapsed since last event \\\hline
num\_processes & integer & Number of processes running  \\\hline
restart\_hours\_elapsed & float & hours elapsed since last restart \\\hline
\end{tabular}
\end{center}

\textbf{\large{rtp\_process\_events}}: RTP Processing events (per obsid)
\begin{center}
 \begin{tabular}{| p{4cm} | p{2cm} | p{10cm} |} 
\hline
 column & type & description \\ [0.5ex]  \hline\hline
\textbf{time} & long & event time in floor(gps seconds) \\ \hline
\textit{\textbf{obsid}} & long integer & observation identifier, foreign key into hera\_obs table \\ \hline
event & string & one of: queued, started, finished, error  \\\hline
\end{tabular}
\end{center}

\textbf{\large{rtp\_process\_record}}: RTP record of processed obsids (entry added when processing finished)
\begin{center}
 \begin{tabular}{| p{4cm} | p{2cm} | p{10cm} |} 
\hline
 column & type & description \\ [0.5ex]  \hline\hline
\textbf{time} & long & record time in floor(gps seconds)\\ \hline
\textit{\textbf{obsid}} & long integer & observation identifier, foreign key into hera\_obs table \\ \hline
pipeline\_list & text & concatenated list of tasks  \\\hline
git\_version & string & git version of RTP code  \\\hline
git\_hash & string & git hash of RTP code  \\\hline
\end{tabular}
\end{center}

\textbf{\large{rtp\_task\_resource\_record}}: RTP record of start and stop times for a task (e.g., omnical) for an obsid, as well as CPU and memory used (if available)
\begin{center}
  \begin{tabular}{| p{4cm} | p{2cm} | p{10cm} |}
\hline
 column & type & description \\ [0.5ex] \hline\hline
\textit{\textbf{obsid}} & long integer & observation identifier, foreign key into hera\_obs table \\ \hline
\textbf{task\_name} & string & name of specific task (e.g., \verb+OMNICAL+) \\ \hline
start\_time & long & start time of task in floor(gps seconds) \\ \hline
stop\_time & long & stop time of task in floor(gps seconds) \\ \hline
max\_mem & float & maximum memory, in MB, consumed by the task; nullable column \\ \hline
avg\_cpu\_load & float & average CPU load, in number of CPUs, for task (e.g., 2.00 means 2 CPUs used); nullable column \\ \hline
\end{tabular}
\end{center}


\subsection{Librarian Tables}
\textbf{\large{lib\_server\_status}}: Librarian version of the server\_status table\\

\textbf{\large{lib\_status}}: High level Librarian status
\begin{center}
 \begin{tabular}{| p{4cm} | p{2cm} | p{10cm} |} 
\hline
 column & type & description \\ [0.5ex]  \hline\hline
\textbf{time} & long & status time in floor(gps seconds) \\ \hline
num\_files & long & total number of files in librarian  \\\hline
data\_volume\_gb & float & total data volume in gigabytes  \\\hline
free\_space\_gb & float & available space in gigabytes  \\\hline
upload\_min\_elapsed & float & minutes elapsed since last file upload \\\hline
num\_processes & integer & number of running background tasks  \\\hline
git\_version & string & git version of Librarian code  \\\hline
git\_hash & string & git hash of Librarian code  \\\hline
\end{tabular}
\end{center}

\textbf{\large{lib\_raid\_status}}: RAID controller status
\begin{center}
 \begin{tabular}{| p{4cm} | p{2cm} | p{10cm} |} 
\hline
 column & type & description \\ [0.5ex]  \hline\hline
\textbf{time} & long & status time in floor(gps seconds) \\ \hline
\textbf{hostname} & string & name of RAID server \\ \hline
num\_disks & int & number of disks in RAID server  \\\hline
info & text & TBD -- various info from megaraid controller, may be several columns \\\hline
\end{tabular}
\end{center}

\textbf{\large{lib\_raid\_errors}}: RAID controller errors/issues
\begin{center}
 \begin{tabular}{| p{4cm} | p{2cm} | p{10cm} |} 
\hline
 column & type & description \\ [0.5ex]  \hline\hline
\textbf{id} & long & auto-incrementing error id\\ \hline
time & long & error report time in floor(gps seconds)\\ \hline
hostname & string & name of RAID server with error \\ \hline
disk & string & name of disk with error \\ \hline
log & text & TBD on format, either a message or a file with the log \\\hline
\end{tabular}
\end{center}

\textbf{\large{lib\_remote\_status}}: Network bandwidth/health to all remote librarians
\begin{center}
 \begin{tabular}{| p{4cm} | p{2cm} | p{10cm} |} 
\hline
 column & type & description \\ [0.5ex]  \hline\hline
\textbf{time} & long & status time in floor(gps seconds)\\ \hline
\textbf{remote\_name} & string & name of remote librarian \\ \hline
ping\_time & float & ping time in seconds \\\hline
num\_file\_uploads & int & number of files uploaded in last 15 minutes  \\\hline
bandwidth\_mps & float & bandwidth to remote in Mb/s, 15 minute average \\\hline
\end{tabular}
\end{center}

\textbf{\large{lib\_files}}: File creation log
\begin{center}
 \begin{tabular}{| p{4cm} | p{2cm} | p{10cm} |} 
\hline
 column & type & description \\ [0.5ex]  \hline\hline
\textbf{filename} & string & name of file created \\ \hline
\textit{obsid} & long integer & observation identifier, foreign key into hera\_obs table. Can be null. \\ \hline
time & long & file creation time in floor(gps seconds)\\ \hline
size\_gb & float & file size in gigabytes \\ \hline
\end{tabular}
\end{center}

\subsection{Correlator Tables}
\textbf{\large{hera\_obs}}: The correlator is the code that will write to the observation table.\\

\textbf{\large{corr\_server\_status}}: Correlator version of the server\_status table\\

\textbf{\large{roach\_temperature}}: Roach (correlator fpga board) temperatures
\begin{center}
 \begin{tabular}{| p{4cm} | p{2cm} | p{10cm} |}
\hline
 column & type & description \\ [0.5ex]  \hline\hline
\textbf{time} & long & measurement time in floor(gps seconds)\\ \hline
\textbf{roach} & string & name of roach (correlator fpga board) \\ \hline
ambient\_temp & float & ambient temperature reported by the roach in degrees C \\\hline
inlet\_temp & float & inlet temperature reported by the roach in degrees C \\\hline
oulet\_temp & float & oulet temperature reported by the roach in degrees C \\\hline
fpga\_temp & float & fpga temperature reported by the roach in degrees C \\\hline
ppc\_temp & float & ppc temperature reported by the roach in degrees C \\\hline
\end{tabular}
\end{center}

The correlator tables are not all defined yet, the following are notes about suggestions and plans for correlator tables. Most of the correlator data will be recorded in a Redis database (a rolling log, ephemeral), that info needs to be grabbed and put in \mc tables.
\begin{outline}[enumerate]
	\1 correlator on/off?	**this is a control**
	\1 Bit statistics (overflows, ADC clipping, bit statistics after bit selects)
	\1 correlator network stats (dropped packets)
	\1 Firmware git hash
	\1 Fengine status
	\1 Xengine status (might be covered in corr\_server\_status)
	\1 Walsh on/off	**this is a control** (correlator propagates to node)
	\1 Noise diode	**this is a control** (correlator propagates to node)
	\1 correlator config (walsh patterns; scaling functions for FFT, bit selection)
	\1 Test mode outputs (results not control) -- very notional
		\2 Fengine sync test
		\2 Xengine test
		\2 Do at beginning and end of night.
		\2 Analog tests
			\3 Noise diode status
			\3 Temperature (i2c device)
			\3 Walsh switching (on/off control. Make sure bit pattern is known and put into data set.)
	\1 SNAP information: all info reported through the correlator
		\2 Feed status
		\2 PAM status

	\1 Node information (from Arduino) (Dave, Jack, Zara, Matt Dexter (mdexter@berkeley.edu), Nima) All node info will be reported through the correlator.
		\2 SNAP power states
		\2 Clock status info -- syncing
		\2 Temperatures (outside + inside, feed?)
		\2 Power PAM, FEM status (binary)
		\2 Node \mc software git hash

These are done:
	\1 \mc information the correlator needs to get and write into files
		\2 Antenna positions
	\1 New info added to correlator files (recorded in hera\_obs table)
		\2 obsid
		\2 duration
\end{outline}

\subsection{QA Info}
This will come from many sources. These are some suggestions for the future, things we might like to see.

\begin{outline}[enumerate]
	\1 RTP/online systems
		\2 RFI statistics/info
		\2 Calibration statistics
		\2 LST repeatability
		\2 TBD other things that come up
	\1 Offline codes (Major work on how to implement this!! Not on the critical path):
		\2 TBD from offline analysis codes
\end{outline}

\subsection{Site Info}

\textbf{\large{weather\_data}}: Weather data from KAT sensors
\begin{center}
 \begin{tabular}{| p{4cm} | p{2cm} | p{10cm} |} 
\hline
 column & type & description \\ [0.5ex]  \hline\hline
\textbf{time} & long & status time in floor(gps seconds)\\ \hline
\textbf{variable} & string & name of weather variable (e.g. wind\_speed,  wind\_direction, temperature) \\ \hline
value & float & value of the variable at this time \\\hline
\end{tabular}
\end{center}

The following are suggestions for the future, things we might like to see.
\begin{outline}[enumerate]
	\1 site power
	\1 network status
\end{outline}

\subsection{Other Future Ideas}
\begin{outline}[enumerate]
	\1 Basic ionospheric monitoring
	\1 RFI monitoring
\end{outline}


\end{document}
